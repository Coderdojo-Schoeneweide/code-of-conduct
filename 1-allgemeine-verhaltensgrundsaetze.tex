\begin{enumerate}
    \item \textit{Respekt}

    Wir schätzen jede:n als individuellen Menschen wert und begegnen einander auf Augenhöhe, mit Respekt und mit Empathie.
    Mobbing und Diskriminierung haben hier keinen Platz.


    \item \textit{Fehlerkultur}

    Das CoderDojo Schöneweide ist ein Raum zum Experimentieren und für verrückte, großartige Ideen.

    Wir helfen und unterstützen einander.

    Fehler sind menschlich -- wir pflegen eine positive Fehlerkultur.
    Das bedeutet, dass wir mit Fehlern konstruktiv umgehen, zu unseren Fehlern stehen und einander verzeihen.

    Wir gehen davon aus, dass jede:r gute Absichten verfolgt und Gründe für das eigene Handeln hat.

    Diskriminierungsfreies und rücksichtsvolles Handeln ist ein Lernprozess.
    Wir unterstützen einander dabei, dazuzulernen, indem wir Probleme offen, konstruktiv und sachlich ansprechen und selber offen für Feedback sind.

    \item \textit{Gesprächskultur}

    Wir unterbrechen einander nicht, außer wir moderieren das Gespräch.

    Wenn wir etwas zum Gespräch beitragen wollen, melden wir uns oder sprechen, wenn es gerade passt.

    Wir sind aufmerksam und beziehen auch Personen mit ein, die zögerlich ihre Meinung äußern.

    Wir bemühen uns, Gespräche nicht übermäßig zu dominieren.

    Jede Meinung und jede Frage ist es wert, gehört zu werden.

    \item \textit{Persönlichkeitsrechte wahren}

    Wir geben vertrauliche und persönliche Informationen nicht ohne Einverständnis aller Beteiligten weiter.

    Wir machen keine Fotos oder Videos von Personen ohne deren Einverständnis und verwenden die Fotos und Videos nur für den vorher abgesprochenen Zweck oder fragen nochmal nach.

    \item \textit{Diversität}

    Wir achten darauf, alle mit einzubeziehen, insbesondere unabhängig ihres Geschlechts, ihres Alters, ihres sozio-ökonomischen Hintergrunds, ihrer sexuellen Orientierung, ihrer politischen Meinung, ihrer Religion, ihrer Sprache, ihrer mentalen und körperlichen Fähigkeiten.

    \item \textit{Nachhaltigkeit}

    Wir sind uns unserer Verantwortung gegenüber der Umwelt bewusst und gestalten unser Engagement nachhaltig.

    \item \textit{Zeitmanagement}

    Wir gehen respektvoll mit der Zeit der anderen um: Wir starten und enden pünktlich oder geben allen Beteiligten vorab rechtzeitig Bescheid.

    Wir schätzen Pausen und Erholung.
\end{enumerate}
