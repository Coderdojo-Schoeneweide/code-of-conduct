\begin{enumerate}
    \item \textit{Kinder- und Jugendschutzplan}

    Wir halten uns an den Kinder- und Jugendschutzplan des CoderDojo Deutschland e.V..

    \item \textit{Verantwortung übernehmen}

    Wir übernehmen Verantwortung für das Wohl der uns anvertrauten Kinder und Jugendlichen und werden das uns Mögliche tun, um sie vor Vernachlässigung, Misshandlung und sexualisierter Gewalt sowie vor gesundheitlicher Beeinträchtigung und vor Diskriminierung jeglicher Art zu schützen.

    Wir nehmen Vorfälle ernst und melden sie nach den Vorgaben des Kinder- und Jugendschutzplan des CoderDojo Deutschland e.V..

    \item \textit{Rechte achten}

    Wir achten auf das Recht der uns anvertrauten Kinder und Jugendlichen auf körperliche Unversehrtheit und Intimsphäre und üben keine Form der Gewalt, sei sie physischer oder psychischer Art, aus.

    \item \textit{Grenzen respektieren}

    Wir respektieren die individuellen Grenzempfindungen der uns anvertrauten Kinder und Jugendlichen und achten darauf, dass auch die Kinder und Jugendlichen diese Grenzen im Umgang miteinander respektieren.
    Wir sind uns bewusst, dass jüngere Kinder ihre Grenzen nicht immer artikulieren können, und sind daher extra achtsam.

    \item \textit{Vorbildfunktion}

    Wir sind uns bewusst, dass wir von den Teilnehmenden als Vorbilder wahrgenommen werden.
    Daher konsumieren wir keine Drogen während und kurz bevor wir ehrenamtlich in Kontakt mit Außenstehenden sind und konsumieren insbesondere auch kein Nikotin in Sichtweite der Teilnehmenden.

    Der Konsens im Team ist, dass momentan Koffein aus kulturellen Gründen von dieser Regel ausgenommen ist.

    \item \textit{Persönliche Entwicklung fördern}

    Wir achten die uns anvertrauten Kinder und Jugendlichen und fördern ihre persönliche Entwicklung.
    Wir leiten sie zu einem angemessenen sozialen Verhalten gegenüber anderen Menschen, zu Respekt, Ehrlichkeit sowie Toleranz an.

    Wir wollen jedem Kind die angemessene Aufmerksamkeit zukommen lassen und bevorzugen keine Kinder.

    Unsere Mentoring-Tätigkeit beschränkt sich auf den Rahmen der CoderDojo\-/Veranstaltungen.
    Private Hilfe ist zu unterlassen oder erfolgt nur nach ausdrücklicher Bitte der Erziehungsberechtigten.

    \item \textit{Umgang mit Fehlverhalten durch Kinder}

    Wir weisen Kinder auf ihr Fehlverhalten hin.
    Wir haben jedoch keinen Erziehungsauftrag, sondern wenden uns bei bestehenden Problemen (z.B.\ verletzendes Verhalten eines Kindes) an die Erziehungsberechtigten der Kinder.
    Wir bestrafen Kinder nicht.

    \item \textit{Altersgerechte Ziele verfolgen}

    Wir richten unser Angebot und unsere Ziele nach dem Entwicklungsstand der uns anvertrauten Kinder und Jugendlichen aus und setzen altersgerechte Lernmethoden ein.

    \item \textit{Persönlichkeitsrechte von Kindern wahren}

    Wenn wir nach Erlaubnis der Teilnehmenden Fotos von ihnen machen, machen wir die Gesichter vor dem Hochladen der Bilder auf Drive unkenntlich und löschen die Bilder anschließend von unseren persönlichen Datenträgern.
\end{enumerate}
